\documentclass{article} % For LaTex2e
\usepackage{iclr2022_conference,times}
% Optional math commands from https://github.com/goodfeli/dlbook_notation.
\input{math_commands.tex}

%######## CHE260: Uncomment your submission name
\newcommand{\chename}{ - Lab Report}
%\newcommand{\chename}{Progress Report}
%\newcommand{\chename}{Final Report}

%######## CHE260: Put your Group Number here
%\newcommand{\gpnumber}{40}

\usepackage{hyperref}
\usepackage{xcolor}
\usepackage[normalem]{ulem}
\usepackage{url}
\usepackage{graphicx}
\usepackage{placeins}
\usepackage{float}
\usepackage{tikz}
\usepackage{multicol}

%######## CHE260: Put your project Title here
\title{Real-Time Neural Signal Filtering via \\
Hodgkin-Huxley Simulation Models}

%######## CHE260: Put your names, student IDs and Emails here
\author{\textbf{Peter Leong} \\
    Student\# 1010892955 \\
    peter.leong@mail.utoronto.ca
\And
    \textbf{Karys Littlejohns} \\
    Student\# 1010893142 \\
    karys.littlejohns@mail.utoronto.ca
\And
    \textbf{Katherine Shepherd} \\
    Student\# 1010895097 \\
    k.shepherd@mail.utoronto.ca
}


% The \author macro works with any number of authors. There are two commands
% used to separate the names and addresses of multiple authors: \And and \AND.
%
% Using \And between authors leaves it to \LaTex{} to determine where to break
% the lines. Using \AND forces a linebreak at that point. So, if \LaTex{}
% puts 3 of 4 authors names on the first line, and the last on the second
% line, try using \AND instead of \And before the third author name.

\newcommand{\fix}{\marginpar{FIx}}
\newcommand{\new}{\marginpar{NEW}}

\iclrfinalcopy 
%######## CHE260: Document starts here
\begin{document}

\maketitle


\begin{abstract}
This project addresses the challenge of automated colourization for 256$\times$256 grayscale images using a dataset of 12,600 image pairs, balanced across human subjects, 
animals, and natural scenery. We frame colourization as a supervised learning problem in the CIELAB colour space, where a model predicts chrominance channels ($a^*$, $b^*$) 
from the luminance channel ($L^*$). A shallow convolutional neural network (CNN) provides the baseline performance, while our primary solution employs a deeper convolutional 
encoder-decoder architecture. This design captures high-level semantic features and spatial context, addressing limitations of shallow networks in perceptual realism.
%######## CHE260: Do not change the next line. This shows your Main body page count.
----Total Pages: \pageref{last_page}
\end{abstract}

\vspace{2ex}

\begin{multicols}{2}

\section{Introduction}
\label{introduction}

\section{Experimental Methods}
\label{experimental_methods}

\subsection{Determining the Mass in the Left Trank}
\label{methods_path_independencies_thermodynamic_properties}

First, the ambient pressure was recorded using the manometer in the lab.
Then, the left tank was pressurized to 40 psi via the following steps:
(1) opening valve A2, (2) opening the left solenoid valve, (3) setting the flow rate to 50 g/min, (4) closing the left solenoid once gauge pressure reached 40 psi, (5) closing valve A2, and (6) setting the flow rate back to 0 g/min.
Then, the tank was left to sit till pressure and temperature stabilized.
Finally, measurements were taken and saved.

\subsection{Determining the Heat Loss in the Left Tank and Specific Heat Capacity}
\label{methods_initial_mass_volume_left_tank}

First, the target temperature was set to $40^{o}C$ by turning on the heaters.
The tank was left to heat up to 40 degrees for five minutes.
Then, the heaters were turned off, and air was run through the tank to allow them to cool down.
This was carried about via the following ordered procedures:
(1) opening the left solenoid valve, (2) opening the "Bar" valve behind the left tank, (3) once the left tank had cooled to \textbf{X degrees celsius}, the left solenoid valve was closed,
(4) the tank was left for air to circulate for 5-10 minutes, and then closed for 1 minute by closing the "Bar" valve, (5) evacuating the left tank by opening the B1 valve, (6) closing the B1 valve.
This process was repeated including \ref{methods_path_independencies_thermodynamic_properties} for tank pressures and temperatures of 70 psig at $40^{o}C$, 40 pisg at $60^{o}C$, and 70 psig at $60^{o}C$.

\section{Results}
\label{results}

\section{Discussion}
\label{discussion}

\section{Conclusion}
\label{conclusion}

% Label placed at the end of main content, inside multicols
\label{last_page}

\end{multicols}  % End the multicolumn environment for main content

% Bibliography starts on new page, outside multicols
\newpage
\bibliographystyle{iclr2022_conference}
\bibliography{CHE260_First_Law_Thermodynamics}

\end{document}