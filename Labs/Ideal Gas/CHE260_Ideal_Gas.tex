\documentclass{article} % For LaTex2e
\usepackage{iclr2022_conference,times}
% Optional math commands from https://github.com/goodfeli/dlbook_notation.
\input{math_commands.tex}

%######## CHE260: Uncomment your submission name
\newcommand{\chename}{ - Lab 1 Report}
%\newcommand{\chename}{Progress Report}
%\newcommand{\chename}{Final Report}

%######## CHE260: Put your Group Number here
%\newcommand{\gpnumber}{40}

\usepackage{hyperref}
\usepackage{xcolor}
\usepackage[normalem]{ulem}
\usepackage{url}
\usepackage{graphicx}
\usepackage{placeins}
\usepackage{float}
\usepackage{tikz}
\usepackage{multicol}
\usepackage{amsmath}
\usepackage{physics}

%######## CHE260: Put your project Title here
\title{Ideal Gas: Path Independecy of  \\
State Functions and Mass Flow Rate}

%######## CHE260: Put your names, student IDs and Emails here
\author{Karys Littlejohns\\
Student\# 1010893142 \\
karys.littlejohns@mail.utoronto.ca
\And
Peter Leong \\
Student\# 1010892955 \\
peter.leong@mail.utoronto.ca 
}

% Reminder to self: put the PRA section somewhere!!

% The \author macro works with any number of authors. There are two commands
% used to separate the names and addresses of multiple authors: \And and \AND.
%
% Using \And between authors leaves it to \LaTex{} to determine where to break
% the lines. Using \AND forces a linebreak at that point. So, if \LaTex{}
% puts 3 of 4 authors names on the first line, and the last on the second
% line, try using \AND instead of \And before the third author name.

\newcommand{\fix}{\marginpar{FIx}}
\newcommand{\new}{\marginpar{NEW}}

\iclrfinalcopy 
%######## CHE260: Document starts here
\begin{document}

\maketitle

\vspace{-3ex}

\begin{abstract}
This project addresses the challenge of automated colourization for 256$\times$256 grayscale images using a dataset of 12,600 image pairs, balanced across human subjects, 
animals, and natural scenery. We frame colourization as a supervised learning problem in the CIELAB colour space, where a model predicts chrominance channels ($a^*$, $b^*$) 
from the luminance channel ($L^*$). A shallow convolutional neural network (CNN) provides the baseline performance, while our primary solution employs a deeper convolutional 
encoder-decoder architecture. This design captures high-level semantic features and spatial context, addressing limitations of shallow networks in perceptual realism.
%######## CHE260: Do not change the next line. This shows your Main body page count.
----Total Pages: \pageref{last_page}
\end{abstract}

\vspace{2ex}

\begin{multicols}{2}


\section{Introduction}
\label{introduction}

\section{Experimental Methods}
\label{experimental_methods}

\subsection{Path Independence of Thermodynamic State Properties}
\label{methods_path_independencies_thermodynamic_properties}

This section describes the experimental procedures for establishing initial pressure conditions in two interconnected tanks and the subsequent methods for achieving thermodynamic equilibrium between them.

\subsubsection{Initial Pressurization Procedure}

The ambient pressure was first recorded using a laboratory manometer. The left tank was pressurized to 40~PSIG through the following sequence: (1)~open valve A2, (2)~activate the left solenoid valve, (3)~set mass flow rate to 50~g/min, (4)~deactivate the left solenoid upon reaching 40~PSIG gauge pressure, (5)~close valve A2, and (6)~reset flow rate to 0~g/min.

The right tank was then evacuated to -6~PSIG using this procedure: (1)~open valves A1 and A5, (2)~activate the right solenoid valve, (3)~set mass flow rate to 50~g/min, (4)~deactivate the right solenoid upon reaching -6~PSIG gauge pressure, (5)~close valves A1 and A5, and (6)~reset flow rate to 0~g/min.

\subsubsection{Equilibrium Method 1: Thermal Equilibrium Focus}

The first equilibrium method utilized the center solenoid valve to prioritize thermal equilibration. The center solenoid valve was opened and maintained until temperature differences between the tanks became negligible. Data acquisition commenced once thermal equilibrium was established, with all digital measurement systems reset prior to recording.

\subsubsection{Equilibrium Method 2: Pressure Equilibrium Focus}

The second equilibrium method employed the micrometer needle valve to emphasize pressure equilibration. Both the small ball valve (B2) and micrometer valve were opened and remained in this configuration until pressure differences between the tanks became insignificant. Experimental measurements were recorded once pressure equilibrium was achieved.
\subsection{Initial Mass \& Volume of Left Tank}
\label{methods_initial_mass_volume_left_tank}

As with part 1, the ambient pressure was recorded using the manometer in the lab.
Then, the left tank was pressurized to a gauge pressure of 40 psi, and was left to wait for pressure and temperature to stabilize.
This was achieved using the following steps:
(1) opening valve A2 and the left solenoid valve, (2) setting the flow rate to 50 g/min, (3) once the gauge pressure reach 40 psi in the left chamber ?, (5) pressure and temperature were left to stabilize.
At this point, the data was measured by the method (whatever method requested by the TA).
Finally, the left tank was emptied using the black pressure release valve (B1).

\section{Results}
\label{results}

\section{Discussion}
\label{discussion}
\subsection{Part 1: Two-Tank Gas Expansion Analysis}

\subsubsection{Volume Ratio Determination}

The volume ratio between the two tanks is determined using mass conservation and the ideal gas law. Both tanks begin at identical initial conditions and reach the same final equilibrium state.

\textbf{Initial State Masses:}
\begin{align*}
m_{1,i} &= \frac{P_i V_1}{R T_i} \\
m_{2,i} &= \frac{P_i V_2}{R T_i}
\end{align*}

\textbf{Final State Masses:}
\begin{align*}
m_{1,f} &= \frac{P_f V_1}{R T_f} \\
m_{2,f} &= \frac{P_f V_2}{R T_f}
\end{align*}

\textbf{Mass Conservation:}
\[
\frac{P_i V_1}{R T_i} + \frac{P_i V_2}{R T_i} = \frac{P_f V_1}{R T_f} + \frac{P_f V_2}{R T_f}
\]

Simplifying yields:
\[
\frac{P_i}{T_i}(V_1 + V_2) = \frac{P_f}{T_f}(V_1 + V_2)
\]

\textbf{Volume Ratio Solution:}
\[
\boxed{\frac{V_1}{V_2} = \frac{m_{1,f}}{m_{2,f}}}
\]

Since both tanks reach identical final pressure and temperature, the volume ratio directly equals the final mass ratio.

\subsubsection{Heat Transfer Analysis}

\textbf{Method 1: Non-Equilibrium Process}\\
When tanks are not maintained in thermal equilibrium, heat transfers occur between the hotter tank, colder tank, and surroundings:
\[
Q_{\text{surr}} + Q_{\text{tank 2}} = -Q_{\text{tank 1}}
\]
Heat flows from the hotter tank (Tank 1) to both the surroundings and the colder tank (Tank 2).

\textbf{Method 2: Constant Thermal Equilibrium}\\
When both tanks are maintained at equal temperatures throughout the process:
\[
Q_{\text{surr}} = -(Q_{\text{tank 1}} + Q_{\text{tank 2}})
\]
Heat transfers directly from the combined system to the surroundings only.

\textbf{State Property Verification}\\
This experiment demonstrates the path independence of state properties. Despite different thermal pathways, both methods reach identical final equilibrium states. The final pressure and temperature depend solely on the initial and final states, not on the intermediate thermal history.

The process can be visualized on temperature-volume diagrams, where Method 1 shows significant pressure variation before thermal equilibration, while Method 2 maintains thermal equilibrium throughout the pressure equalization.

\subsection{Part 2: Initial Mass and Volume Determination}

\subsubsection{Theoretical Framework}

The initial mass and volume of the tank are determined using mass conservation and the ideal gas law:

\begin{align}
M_f &= M_i + \int \dot{m}  dt \label{eq:mass_cons} \\
P_i V &= M_i R T_i \label{eq:ig_init} \\
P_f V &= M_f R T_f \label{eq:ig_final}
\end{align}

\subsubsection{Solution Methodology}

Eliminating the unknown volume $V$ by dividing the initial and final ideal gas equations:
\[
\frac{P_i}{P_f} = \frac{M_i T_i}{M_f T_f}
\]

Substituting the mass conservation relation:
\[
\frac{P_i}{P_f} = \frac{M_i T_i}{(M_i + \int \dot{m}  dt) T_f}
\]

Solving for the initial mass $M_i$:
\[
\boxed{M_i = \frac{-P_i T_f \int \dot{m}  dt}{P_i T_f - P_f T_i}}
\]

The tank volume is subsequently determined from the ideal gas law:
\[
V = \frac{M_i R T_i}{P_i}
\]

\subsubsection{Experimental Uncertainties}

Several experimental uncertainties affected the measurement accuracy. Achieving precisely 40 psig initial pressure introduced human error during the pressurization process. Instrument resolution limitations created inherent measurement uncertainties in both pressure and temperature readings. The mass flow meter possessed finite precision that propagated through the mass integration calculation. Thermal equilibrium assumptions may not have been perfectly satisfied during data collection, introducing additional systematic error.

\subsubsection{Ideal Gas Law Assumptions}

The application of the ideal gas law introduces theoretical approximations that affect results. The model assumes gas molecules have negligible volume, possess no intermolecular forces, and undergo perfectly elastic collisions. In reality, air exhibits compressibility effects that deviate from ideal behavior. However, these deviations remain minimal under the experimental conditions of 40 psig and approximately 25°C. At these moderate pressure and temperature levels, the compressibility factor approaches unity, validating the ideal gas law as an appropriate engineering approximation for this analysis.
\section{Conclusion}
\label{conclusion}

\label{last_page}

\newpage
\bibliographystyle{iclr2022_conference}
\bibliography{CHE260_Proposal_Ref}
    
\end{multicols}

\end{document}