\documentclass{article} % For LaTex2e
\usepackage{iclr2022_conference,times}
% Optional math commands from https://github.com/goodfeli/dlbook_notation.
\input{math_commands.tex}

%######## CHE260: Uncomment your submission name
%\newcommand{\chename}{ - Project Proposal}
%\newcommand{\chename}{Progress Report}
%\newcommand{\chename}{Final Report}

%######## CHE260: Put your Group Number here
%\newcommand{\gpnumber}{40}

\usepackage{hyperref}
\usepackage{xcolor}
\usepackage[normalem]{ulem}
\usepackage{url}
\usepackage{graphicx}
\usepackage{placeins}
\usepackage{float}
\usepackage{tikz}
\usepackage{multicol}

%######## CHE260: Put your project Title here
\title{Real-Time Neural Signal Filtering via \\
Hodgkin-Huxley Simulation Models}

%######## CHE260: Put your names, student IDs and Emails here
\author{Karys Littlejohns\\
Student\# \\
karys.littlejohns@mail.utoronto.ca
\And
Peter Leong \\
Student\# 1010892955 \\
peter.leong@mail.utoronto.ca \\
\AND
}

% The \author macro works with any number of authors. There are two commands
% used to separate the names and addresses of multiple authors: \And and \AND.
%
% Using \And between authors leaves it to \LaTex{} to determine where to break
% the lines. Using \AND forces a linebreak at that point. So, if \LaTex{}
% puts 3 of 4 authors names on the first line, and the last on the second
% line, try using \AND instead of \And before the third author name.

\newcommand{\fix}{\marginpar{FIx}}
\newcommand{\new}{\marginpar{NEW}}

\iclrfinalcopy 
%######## CHE260: Document starts here
\begin{document}

\maketitle

\vspace{-6ex}

\begin{abstract}
This project addresses the challenge of automated colourization for 256$\times$256 grayscale images using a dataset of 12,600 image pairs, balanced across human subjects, 
animals, and natural scenery. We frame colourization as a supervised learning problem in the CIELAB colour space, where a model predicts chrominance channels ($a^*$, $b^*$) 
from the luminance channel ($L^*$). A shallow convolutional neural network (CNN) provides the baseline performance, while our primary solution employs a deeper convolutional 
encoder-decoder architecture. This design captures high-level semantic features and spatial context, addressing limitations of shallow networks in perceptual realism.
%######## CHE260: Do not change the next line. This shows your Main body page count.
----Total Pages: \pageref{last_page}
\end{abstract}

\vspace{2ex}

\begin{multicols}{2}

\section{Introduction}
\label{introduction}

\section{Experimental Methods}
\label{experimental_methods}

\subsection{Path Independencies of Thermodynamic State Properties}
\label{methods_path_independencies_thermodynamic_properties}

This section outlines the main process to bring the two tanks up to the differing pressures as outlined by the experiment.
Additionally, the two different methods to reach equilibrium between the two tanks are outlined afterward.

\subsubsection{Main Procedure}

The ambient pressure was recorded first using the manometer in the lab.
Then, the left tank was pressurized to 40 PSI via the following steps:
(1) opening valve A2, (2) opening the left solenoid valve, (3) setting the flow rate to 50 g/min, (4) closing the left solenoid once gauge pressure reached 40 psi, (5) closing valve A2, and (6) setting the flow rate back to 0 g/min.
Afterwhich, the right tank was pressured to -6 psi through:
(1) opening valves A1 and A5, (2) opening the right solenoid valve, (3) setting the flow rate to 50 g/min, (4) closing the right solenoid once gauge pressure reach -6 psi, (5) closing valves A2 and A5, and (6) setting the flow rate back to 0 g/min.

\subsubsection{Equilibrium Method 1}

Method 1 of reaching equilibrium primarily used the center solenoid valve.
First, the center solenoid valve was opened until there was an insignificant temperature difference between the two tanks.
Measurements were then taken after this point, and the digital readings were reset.

\subsubsection{Equilibrium Method 2}

Method 2 of reaching equilibrium used the micrometer needle valve.
The small ball valve (B2) and the micrometer valve were opened.
Both valves were left open until there was an insignificant pressure difference between the two tanks, upon which measurements were recorded.

\subsection{Initial Mass \& Volume of Left Tank}
\label{methods_initial_mass_volume_left_tank}

As with part 1, the ambient pressure was recorded using the manometer in the lab.
Then, the left tank was pressurized to a gauge pressure of 40 psi, and was left to wait for pressure and temperature to stabilize.
This was achieved using the following steps:
(1) opening valve A2 and the left solenoid valve, (2) setting the flow rate to 50 g/min, (3) once the gauge pressure reach 40 psi in the left chamber ?, (5) pressure and temperature were left to stabilize.
At this point, the data was measured by the method (whatever method requested by the TA).
Finally, the left tank was emptied using the black pressure release valve (B1).

\section{Results}
\label{results}

\section{Discussion}
\label{discussion}

\section{Conclusion}
\label{conclusion}

\label{last_page}

\newpage
\bibliographystyle{iclr2022_conference}
\bibliography{CHE260_Proposal_Ref}

\end{multicols}
\end{document}